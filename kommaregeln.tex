% XeTeX

\documentclass[landscape]{article}

% paper geometry
\usepackage[landscape]{geometry}
\geometry{
    a4paper,
    total={210mm,297mm},
    left=15mm,
    right=15mm,
    top=15mm,
    bottom=20mm
}

% header and footer
\usepackage{fancyhdr}
\pagestyle{fancy}
\renewcommand{\headrulewidth}{0pt}
\renewcommand{\footrulewidth}{0pt}
\fancyfoot[LO, RE] {Kommaregeln}
\fancyfoot[C]{-\,\thepage\,-}

% including images
\usepackage{graphicx}
\graphicspath{{images/}}

% columns
\usepackage{multicol}
\columnseprule0.5pt
\columnsep5mm

% german spelling
\usepackage{fontspec}
\usepackage{polyglossia}
%\usepackage[ngerman]{babel}
\setdefaultlanguage{german}

% subsubsection spacing
\usepackage{titlesec}
\titlespacing*{\section}{0pt}{1em}{0.4em}
\titlespacing*{\subsection}{0pt}{1em}{0.4em}
\titlespacing*{\subsubsection}{0pt}{1em}{0.4em}

% paragraph title format
%\titleformat{\paragraph}{\normalfont\scshape}{\theparagraph}{}{}
%\titlespacing*{\paragraph}{0pt}{1em}{0.2em}

% referencing section names
\usepackage{hyperref}
\makeatletter
\def\ttl@useclass#1#2{%
  \@ifstar
    {\ttl@labelfalse\@dblarg{#1{#2}}}% {\ttl@labelfalse#1{#2}[]}%
    {\ttl@labeltrue\@dblarg{#1{#2}}}}
\makeatother

% own reference command
\newcommand{\ownref}[1]{(siehe: \nameref{#1}, S. \pageref{#1})}

% tiny break
\newcommand{\tbreak}{\\\vspace{-3mm}}

% german quotation marks
\newcommand*\glqq{„}
\newcommand*\grqq{“}
\newcommand{\gqm}[1]{\glqq #1\grqq}

% z.B.
\newcommand{\zB}{z.\,B.}

% turnoff automatic indentation
\setlength{\parindent}{0pt}

% enumerate
%\usepackage{enumerate,letltxmacro}
%\LetLtxMacro\itemold\item
%\renewcommand{\item}{\itemindent-2mm\itemold}

\usepackage{enumitem}
\setlist[itemize]{
    labelsep=0.3em,
    labelindent=0.5\parindent,
    itemindent=0pt,
    itemsep=-0.3em, 
    leftmargin=*,
    before=\vspace{-6mm},
    after=\vspace{-3mm}
}

\setlist[enumerate]{
    labelsep=0.3em,
    labelindent=0.5\parindent,
    itemindent=0pt,
    itemsep=-0.3em, 
    leftmargin=*,
    before=\vspace{-6mm},
    after=\vspace{-3mm}
}


\begin{document}
\begin{multicols*}{3}

    \section*{Kommaregeln}

    \textbf{1. Das Komma steht zwischen Aufzählungen gleichartiger Satzglieder.}\tbreak

    \textsc{Beispiele}\\

    \begin{itemize}
        \item Meine Freundin ist ein hübsches, liebneswertes, intelligentes Mädchen.
        \item Sie liebt Musik, schicke Kleider \textit{und} schnelle Autos. {\footnotesize (Hier hat \gqm{und} das Komma ersetzt)}
    \end{itemize}

    \textsc{Beachte}\tbreak

    Das Komma wird ersetzt durch die Wörter:\\
    und, oder, sowie, wie, beziehungsweise / bzw., \mbox{sowohl \dots\, als auch}, \mbox{entweder \dots\, oder}, \mbox{weder \dots\, noch}\\

    \textbf{2. Das Komma steht vor entgegengesetzten Konjunktionen}\tbreak

    \zB: aber, sondern, allein, doch, jedoch, vielmehr\tbreak

    \textsc{Beispiele}\\
    
    \begin{itemize}
        \item Ihr Vater war ein grober, aber gutmütiger Kerl.
        \item Nicht nur seine Hände, sondern auch seine Füße waren rießig.
    \end{itemize}\quad\tbreak

    \textbf{3. Das Komma steht nach Anreden}\tbreak

    Eingeschobene Anreden, werden durch ein Komma davor und danach abgegrenzt.\tbreak

    \textsc{Beispiele}\\

    \begin{itemize}
        \item Herr Lehrer, ich bin gut vorbereitet!
        \item Lieber Michael, ich schreibe dir \dots
        \item Dir, lieber Vater, gratuliere ich \dots\, {\footnotesize(eingschoben)}
        \item Ich beglückwünsche dich, lieber Hans-Peter, zur bestandenen Prüfung. {\footnotesize(eingschoben)}
    \end{itemize}\quad\tbreak

    \textbf{4. Das Komma steht nach Empfindungswörtern, wenn sie hervorgehoben werden}\tbreak
    
    \textsc{Beispiele}\\

    \begin{itemize}
        \item Oh je, war das eine Arbeit!
        \item Verflixt, schon wieder eine Sechs!
        \item Aua, du tust mir weh!
    \end{itemize}

    \textsc{Beachte}\tbreak

    Ohne Hervorhebung steht kein Komma, \zB:\\ 
    Ach lass mich in Ruhe! Oh wenn sie doch käme!\\

    \textbf{5. Das Komma schließt Appositionen ein}\tbreak

    \textsc{Beispiele}\\

    \begin{itemize}
        \item Der Direktor, ein alter Fuchs, lächelte.
        \item Frau Müller, die Schulsekretärin, ist immer bestens informiert.
    \end{itemize}\quad\tbreak

    \textbf{6. Das Komma schließt Erläuterungen ein, die durch \gqm{d.\,h.}, \gqm{nämlich}, \gqm{\zB}, \gqm{wie}, \gqm{und zwar} eingeleitet werden}\tbreak

    \textsc{Beispiele}\\

    \begin{itemize}
        \item An einem Tag war der Biologieunterricht besonders interessant, nämlich am Freitag.
        \item Bestimmte Themen, \zB\, Balzverhalten und Fortpflanzung, interessieren uns besonders.
    \end{itemize}\quad\tbreak

    \textbf{7. In Satzreihen werden Hauptsätze durch Kommata getrennt} (1)\tbreak

    Das Komma steht auch, wenn ein Hauptsatz in einen anderen eingeschoben wird. (2)\\
    Werden zwei vollständige Hauptsätze durch \gqm{und} bzw. \gqm{oder} verbunden, \textit{kann} das Komma stehen. (3) und (4)\tbreak

    \textsc{Beispiele}\\

    \begin{enumerate}[label=(\arabic*)]
        \item Er rannte in den Klassenraum, er sah sich um, er handelte.
        \item Du kannst, ich betone es noch einmal, nicht an dieser Schule bleiben
        \item Er rief den Schüler zu sich, und dieser nahm sein Zeugnis entgegen.
        \item Er rief den Schüler zu sich und dieser nahm sein Zeugnis entgegen.
    \end{enumerate}\quad\tbreak

    \textbf{8. Das Komma steht zwischen Satzteilen, die durch anreihende Konjunktionen in der Art einer Aufzählung verbunden sind}\tbreak

    \zB\, \mbox{bald -- bald}, \mbox{einerseits -- andererseits}, \mbox{teils -- teils}, \mbox{je -- desto}, \mbox{ob -- ob}, \mbox{halb -- halb}, \mbox{nicht nur -- sondern auch}\tbreak

    \textsc{Beispiele}\\
   
    \begin{itemize}
        \item Einerseits verhält sich Susi noch wie ein kleines Mädchen, andererseites möchte sie gern schon erwachsen sein.     
        \item Teils spielt sie mit ihren alten Puppen, teils schminkt sich sich wie ein Model.
        \item Ob sie mit Puppen spielt, ob sie sich schminkt -- süß ist sie allemal.
    \end{itemize}\quad\tbreak

    \textbf{9. Das Komma trennt den Nebensatz vom übergeordneten Hauptsatz ab}\\
    \begin{enumerate}[label=(\alph*)]
        \item Kausal-, Temporal-, Konditional-, Konzessiv-, Konsekutiv- , Final- und Modalsatz\\
            \textsc{Beispiele}\\

            \begin{itemize}
                \item Weil es klingelt, gehen die Schüler in ihren Klassenraum.
                \item Die Schüler gehen in ihren Klassenraum, weil es klingelt.
                \item Die Schühler gehen, weil es klingelt, in ihren Klassenraum.
            \end{itemize}\quad\vspace{-4mm}

        \item indirekter Fragesatz\\
            \textsc{Beispiele}\\

            \begin{itemize}
                \item Niemand wusste, wann die nächste Klassenarbeit geschrieben werden sollte.
                \item Wann die nächste Klassenarbeit geschrieben werden sollte, wusste niemand.
            \end{itemize}\quad\vspace{-4mm}

        \item Relativsatz\\
            \textsc{Beispiele}\\

            \begin{itemize}
                \item Die junge Dame, die du vorstellen willst, kenne ich schon.
                \item Ich kenne die junge Dame, die du mir vorstellen willst.
            \end{itemize}
    \end{enumerate}\quad\tbreak

    \textbf{10. Das Komma steht zwischen Aufzählungen gleichartiger Nebensätze, wenn diese nicht durch \gqm{und} bzw. \gqm{oder} verbunden sind.}\tbreak
        
    \textsc{Beispiel}\\
    Weil sie hübsch ist, weil sie mich liebt \textit{und} weil sie zudem einen reichen Vater hat, werde ich sie heiraten.\\

    \textbf{11. Das Komma steht nach herausgehobenen Satzteilen, die durch ein Pronomen oder Adverb erneut aufgenommen werden}\tbreak

    \textsc{Beispiele}\\

    \begin{itemize}
        \item Deine Schwester, die habe ich gut gekannt.
        \item In meiner Studentenbude, da haben wir uns oft geküsst.
    \end{itemize}\quad\tbreak

    \textbf{12. Erweiterte Infinitive grenzt man durch Kommata ab, wenn}\\

    \begin{enumerate}[label=(\alph*)]
        \item die Infinitvgruppe durch \gqm{um}, \gqm{ohne}, \gqm{statt}, \gqm{anstatt}, \gqm{außer}, \gqm{als} eingeleitet wird.\\
            \textsc{Beispiele}\\

            \begin{itemize}
                \item Sie gab mir einen Kuss, \textit{um} mich damit um Verzeihung zu bitten.
                \item Er fuhr los, \textit{ohne} auf die rote Ampel zu achten.
                \item Ihr fiel nichts Besseres ein, \textit{als} zu lügen.
            \end{itemize}\quad\vspace{-4mm}

        \item die Infinitivgruppe von einem Substantiv abhängt.\\
            \textsc{Beispiele}\\

            \begin{itemize}
                \item Ihm wurde bei dem \textit{Gedanken}, morgen eine Klassenarbeit \textit{zu schreiben}, heiß und kalt.
                \item Er wurde bei dem \textit{Versuch}, das Geld \textit{zu stehlen}, vom Klassenlehrer beobachtet.
                \item Sie fasste den \textit{Plan}, heimlich \textit{abzureisen}.
            \end{itemize}\quad\vspace{-4mm}

        \item die Infinitivgruppe von einem Verweiswort abhängt.\\
            \textsc{Beispiele}\\

            \begin{itemize}
                \item Peter liebt \textit{es}, abends in einem Buch zu lesen.
                \item \textit{Es} gefällt mir, wie du dich anziehst.
                \item Herbert hat \textit{es} nie versäumt, mir zum Geburtstag zu gratulieren.
            \end{itemize}
    \end{enumerate}\quad\tbreak

    \textbf{13. Das Komma kann das erweiterte Partizip vom Satz trennen. Ist das erweiterte Partizip in den Satz eingeschoben oder nachgestellt, muss es durch Kommata abgetrennt werden.}\tbreak
    
    \textsc{Beispiel}\\

    \begin{itemize}
        \item Vor Angst zitternd, stand der Übeltäter da.
        \item Vor Angst zitternd stand der Übertäter da.
    \end{itemize}\quad\vspace{-4mm}

    \textsc{Beachte}\\
    
    \begin{itemize}
        \item Der Direktor, verägert durch den Lärm, eilte herbei.
        \item Die Sportler standen in der Halle, in Reih und Glied angetreten.
    \end{itemize}\quad\tbreak

    \textbf{14. Das Komma trennt zwei ungebeugte Partizipien vom Satz, wenn diese durch \gqm{und} verbunden sind}\tbreak

    \textsc{Beispiele}\\

    \begin{itemize}
        \item Der Deutschlehrer, geachtet und geliebt, betrat den Klassenraum.
        \item Die Schüler, ächzend und stöhnend, schrieben eine Klassenarbeit.
    \end{itemize}\quad\tbreak

    \textbf{15. Das Komma trennt zwei nachgestellte Adjektive vom Satz, wenn diese durch \gqm{und} verbunden sind}\tbreak

    \textsc{Beispiele}\\

    \begin{itemize}
        \item Alle Schüler, große und kleine, fürchten sich vor einer Sechs.
        \item Die Sonne, hell und klar, ging über ihnen auf.
    \end{itemize}\quad\tbreak

    \textbf{16. Das Komma gliedert mehrteilige Datums- und Zeitangaben}\tbreak

    \textsc{Beispiele}\\
    \begin{itemize} 
        \item Tübingen, den 28. Mai 2016
        \item Uppsala, im Februar 2016
        \item Ich komme am Samstag, den 12. Dezember, (um) 18.30 am Stuttgarter Hauptbahnhof an.
    \end{itemize}

    \section*{Quelle}
    Udo Klinger, \url{http://www.udoklinger.de/Deutsch/Grammatik/Kommaregeln.htm}, Schwerte/Ruhr 2012


\end{multicols*}
\end{document}
